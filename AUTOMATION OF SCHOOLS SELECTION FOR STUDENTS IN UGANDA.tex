\documentclass{article}


\begin{document}

\begin{center}
{\huge\textbf{AUTOMATION OF SCHOOLS \\ 
			SELECTION FOR STUDENTS \\ 
			IN UGANDA \\ }}

\end{center} 



\newpage

\section{INTRODUCTION}.

In Uganda’s education system, there are three major qualifications one has to attain before making it to the University which are, Primary Leaving Examinations \textbf{(P.L.E)}, Uganda Certificate of education \textbf{(U.C.E)} and Uganda Advanced Certificate of Education \textbf{(U.A.C.E)}. A student studies for seven, four and two years respectively before moving onto the next level. However, one usually has to attend to a nursery school or kindergarten before joining the primary level of education. The nursery level has no qualification attained, which gives some parents reason to make their children skip this level. \\

At the end of every level, students have an opportunity to join another school where they can study the next level after doing final exams at the current level. These examinations are always sat for annually as the trend has been for a number of years. These exams are set and controlled by a body known as Uganda National Examination Board \textbf{(UNEB)}. \cite{article} The systems of examination format and grading used by UNEB was inherited from University of Cambridge Local Examination Syndicate although this has changed over time. The syndicate was responsible for school examinations in the British colonies until 1968 when the East African Examination Council was formed. It was in 1983 that UNEB became active when the P.L.E section was transferred to them.  \\ 

The process of selecting the school to join is quite long and bureaucratic. Above all, this process is done manually by teachers as shown in the Observer newspaper of March 27\textsuperscript{th}, 2017.\cite{article2} One big problem associated with this process is bribery where some parents buy slots in schools for their children who did not perform to the standards required by that particular school. \\

Although the UNEB curriculum is the most widely recognized curriculum in Uganda, it is important to note that there are schools which offer curriculums of foreign countries for example United States of America, Great Britain etc. For this research topic however, we’ll be focussing on the curriculum offered by UNEB. \\

\begin{thebibliography}{10}

\bibitem{article} Uganda National Examinations Board (2016).  
\emph{Historical Background} [article], 
Available:  \texttt{http://uneb.ac.ug/index.php/our-history/}

\bibitem{article2} Christian Basl, The Observer (2017, Mar. 27). 
\emph(Secondary school selection exercise: who's in control?) [article], 
Available: \texttt{http://observer.ug/education/51981-secondary-schools-selection-exercise-who-is-in-control.html}



\end{thebibliography}



\end{document}
